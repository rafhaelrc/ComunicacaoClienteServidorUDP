	\documentclass{abnt}
%\usepackage[a4paper, inner=1.5cm, outer=2cm, top=3cm, bottom=2cm, bindingoffset=1cm]{geometry}
\usepackage[utf8]{inputenc}
\usepackage[english,brazilian]{babel}
\usepackage{rotating}
\usepackage{hyperref}
\usepackage{url}
\usepackage{indentfirst}
\hypersetup{%
    pdfborder = {0 0 0}
}
\usepackage{graphics}
\graphicspath{{./figuras/}}
\usepackage{placeins}
%\usepackage{lscape}
\usepackage{pdflscape}

\begin{document}

\autor{Rafhael Rodrigues Cunha\par Marcelo Maia Lopes }

\titulo{Redes e Sistemas Distribuídos: Trabalho Prático II}

%\orientador{Prof.}

\instituicao{Universidade Federal do Pampa \par Engenharia da Software}

\local{Alegrete - RS, Brasil}

\data{20 de Maio de 2012}

\capa

\folhaderosto

\tableofcontents


\chapter{Introdução}
Esse Documento tem por objetivo auxiliar o usuário para rodar a aplicação cliente/servidor UDP que desenvolvemos.

\clearpage
\chapter{Manual do Usuário}

	\section{Instação}	
	Para utilizar o cliente servidor UDP que desenvolvemos é necessario pegar a pasta do projeto que vai estar contida no zip no qual vai ser enviado como entrega final do Trabalho. Logo após isso é necessario vincular o mesmo a uma IDE de programação que suporte a linguagem java (de preferência o Netbeans) e abrir duas classes:
	Após abrir a classe TCPServer (o nome da classe é irrelevante estamos utilizando o protocolo UDP para implementar a mesma), faremos a seguinte alteração:
		\begin{itemize}
			\item DatagramSocket serverSocket = new DatagramSocket(2222); Esse 2222 entre parenteses refere-se a porta na qual eu gostaria que meu server roda-se. Caso essa não seja a sua escolhida, e só trocar ali.
			\item Após essa pequena de alteração de porta, e só rodar a classe do servidor.
		\end{itemize}	
	Para realmente começarmos a utilizar o sistema precisamos configurar nossa classe ClienteUDP	
		Para isso na seguinte linha iremos colocar o ip do nosso server:
		\begin{itemize}
		\item InetAddress IPAddress = InetAddress.getByName("201.66.204.109"); Lembrando que se deseja testar em sua própria maquina substitua o que esta entre " " por localhost.
		\end{itemize}	
		Após, vamos alterar a seguinte linha: 
		
		\begin{itemize}
				\item new DatagramPacket(sendData, sendData.length, IPAddress, 2222); Onde o 2222 é a porta na qual efetuarei comunicação com o meu servidor (lembro que configuramos isso na hora de rodar o servidor.)
				\end{itemize}	
	
	Após essas configurações, rodamos a classe ClienteUDP, informamos o número de requisições que a mesma deseja fazer ao servidor e logo recebemos como resposta na tela a impressão da que demorou menos tempo para ser concluída.
	
	\section{Linguagem Utilizada}	
		A linguagem utilizada de apoio para desenvolvimento da solução do trabalho proposto foi o java, porque alem de ele ser multiplataforma, ou seja, rodar em diversos sistemas operacionais ele também possui muitos tratamentos na parte de sockets que é uma classe utilizada na implementação da solução que facilita na parte de comunicação entre um cliente e outro. \cite{ORACLE}
		
	



\clearpage
%Referências Bibliograficas
\nocite{*}
\bibliographystyle{plain}		
\bibliography{bibliografia}	
	
\end{document}